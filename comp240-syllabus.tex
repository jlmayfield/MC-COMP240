\documentclass[10pt]{article}
\usepackage{amsmath}
\usepackage{setspace}
\usepackage{hyperref}
\usepackage{booktabs}

\setlength{\textheight}{9in} \setlength{\topmargin}{-.5in}
\setlength{\textwidth}{6.5in} \setlength{\oddsidemargin}{0in}
\setlength{\evensidemargin}{0in}

\title{Syllabus \\ COMP 240 \\ Computer Applications}
\author{  }
\date{Spring 2020}

\begin{document}
\maketitle

\section{Logistics}
\begin{itemize}
\item \textbf{Where: } Center for Science and Business (CSB), Room 323
\item \textbf{When: } TTh 2--3:40pm
\item \textbf{Instructor: } Logan Mayfield
\begin{itemize}
\item \textit{Office: } Center for Science and Business (CSB), Room 344
\item \textit{Phone: } 309-457-2200 % chktex 8
\item \textit{Website: } \url{http://jlmayfield.github.io/}
\item \textit{Email: } lmayfield \textit{at} monmouthcollege \textit{dot} edu
\item \textit{Office Hours: }  M 1:30--3pm, Tu 9--10am, Th 1--2pm, and by appointment.
\end{itemize}
\item \textbf{Website: } \url{http://jlmayfield.github.io/teaching/COMP240/}
\item \textbf{Credits: } 1 course credit
\item \textbf{Prerequisites: } A C or better in COMP 151 and COMP 152.
\end{itemize}
\emph{Note: This Syllabus is subject to change based on specific class needs. Significant deviations from the syllabus will be discussed in class.}

\section{Description, Content, and Learning Goals}

In Computer Applications students will work in small groups to develop three different computer applications.  Each application will expose them to a different computing platform along with the tools and computing concepts used in developing programs for that platform. The platform and purpose of each applications will vary from year to year and instructor to instructor, but common choices of platforms include: the command line interface, the web, mobile devices, and high-performance computing. Students will maintain and develop their projects using GitHub and Git version control software. Students will also engage in peer-review of the work of their team members and when possible the other development teams in the course. Upon completing the course students will know how to apply basic software engineering practices in a small group setting, how to maintain software through the git version control system, and will have experience with tools and best-practices for developing modern software applications for three different computing platforms.

\subsection{Textbook}

Books and reference materials will be based on projects assigned but are likely to be a combination of online resources and instructor provided handouts.  Students should consult the course website for project-by-project materials.

\section{Workload}
% number of/details on midterms, finals, project, homeworks, quizes, etc

The course workload is as follows:
\begin{center}
  \begin{tabular}{ll}
    \underline{Category} & \underline{Number of Assignments} \\
    Presentations & 6 \\
    Peer-Reviews & 6 \\
    Projects & 3 \\
  \end{tabular}
\end{center}

You can expect to spend most class meetings checking in with your current development team and the course instructor. Accompanying each project there will be two presentations: one checkpoint presentation and one final presentation. These presentations will take place during scheduled class times as well. You will carry out peer-reviews and self-evaluations after each presentation.

\subsection{Course Engagement Expectations}

The weekly workload for this course will vary by student but on average should be about 12 to 13 hours per week.  The follow tables provides a rough estimate of the distribution of this time over different course components.
\begin{center}
\begin{tabular}{ll}
\underline{Assignment Type} & \underline{Time/week} \\
Class Meetings       & 3.3 hours/week \\
Project Work          & 6-7 hours/week \\
Presentations/Peer-Review   & 2 hours/week \\
\bottomrule
 & ~12.5 hours/week
\end{tabular}
\end{center}

\section{Grades}

This course uses a standard grading scale where percentage grades translate to letter grades as follows:

\begin{center}
\begin{small}
\begin{tabular}{lcl}
\underline{Score} & & \underline{Grade} \\
94--100 & & A \\
90--93 & & A- \\
88--89 & & B+ \\
82--87 & & B \\
80--81 & & B- \\
78--79 & & C+ \\
72--77 & & C \\
70--71 & & C- \\
68--69 & & D+ \\
62--67 & & D \\
60--61 & & D- \\
0--59 & & F
\end{tabular}
\end{small}
\end{center}


You are always welcome to challenge a grade that you feel is unfair or calculated incorrectly.  Mistakes made in your favor will never be corrected to lower your grade.  Mistakes made not in your favor will be corrected.  \textit{Basically, after the initial grading, your score can only go up as the result of a challenge.}


\subsection{Grade Weights}

Your final grade is based on a weighted average of presentation scores, projects, and overall participation in the course.

\begin{center}
  \begin{tabular}{ll}
  \underline{Category} & \underline{Weight} \\
    Presentations & 35\% \\
    Projects & 45\% \\
    Participation & 15\% \\
    Final Self-Evaluation & 5\%
  \end{tabular}
\end{center}

Your individual project and presentation grades will be determined based on the overall group effort as well as your individual contributions to the application code. Individual contributions will be assessed through the project tracking features on GitHub, through feed-back provided by peer-reviews done about your work, and through your own self-evaluations. It will not necessarily be the case that each member of a group receive the same grade on a project or presentation.  Participation grades will be determined by class attendance, contributions to discussions on GitHub, and through the quality of your peer-reviews submitted about other members of the class. During the final exam period you will be required to carry out one last self-evaluation and self-reflection about the work you did throughout the course of the semester.


\subsection{Attendance}

Unexcused absences will have a detrimental effect on the participation component of your grade. Having regular face-to-face time with your group is vital to the success of the project. If you must miss class, then make every possible effort to notify the instructor and your development group members of your absence before it occurs.

\subsection{Academic Honesty}

From the Monmouth College Academic Honesty Policy:
\begin{quote}
  ``We view academic dishonesty as a threat to the integrity and intellectual mission of our institution. Any breach of the academic honesty policy - either intentionally or unintentionally - will be taken seriously and may result not only in failure in the course, but in suspension or expulsion from the college. It is each student’s responsibility to read, understand and comply with the general academic honesty policy at Monmouth College, as defined here in the Scots Guide, and to the specific guidelines for each course, as elaborated on the professor’s syllabus.''

  ``The following areas are examples of violations of the academic honesty policy:
  \begin{enumerate}
  \item Cheating on tests, labs, etc;
  \item Plagiarism, i.e., using the words, ideas, writing, or work of another without giving appropriate credit;
  \item Improper collaboration between students, i.e., not doing one’s own work on outside assignments specified as group projects by the instructor;
  \item Submitting work previously submitted in another course, without previous authorization by the instructor.''
  \end{enumerate}

  ``Please note that this list is not intended to be exhaustive.''
\end{quote}

The complete Monmouth College Academic Honesty Policy can be found on the College web page by clicking on ``Student Life'' then on ``Scot’s Guide'' in the navigation bar to the left, then ``Academic Regulations'' in the navigation bar at the left.  Or you can visit the web page directly by typing in this URL: \url{https://ou.monmouthcollege.edu/life/residence-life/scots-guide/academic-regulations.aspx}

In this course, any violation of the academic honesty policy will have varying consequences depending on the severity of the infraction as judged by the instructor. Minimally, a violation will result in an``F'' or 0 points on the assignment in question. Additionally, the student’s course grade may be lowered by one letter grade. In severe cases, the student will be assigned a course grade of ``F'' and dismissed from the class. All cases of academic dishonesty will be reported to the Associate Dean who may decide to recommend further action to the Admissions and Academic Status Committee, including suspension or dismissal. It is assumed that students will educate themselves regarding what is considered to be academic dishonesty, so excuses or claims of ignorance will not mitigate the consequences of any violations.

\section{Accessibility}

Student Success \& Accessibility Services offers FREE resources to assist Monmouth College students with their academic success. Programs include Supplemental Instruction for select classes, Drop-In and appointment tutoring, and individual Academic Coaching. Our office is here to help all students excel academically, since all students can work toward better grades, practice stronger study skills, and manage their time better.

If you have a disability or had academic accommodations in high school or another college, you may be eligible for academic accommodations at Monmouth College under the Americans with Disabilities Act (ADA). Monmouth College is committed to equal educational access. To discuss any of the services offered, please call or meet with Robert Crawley, Interim Director of Student Success \& Accessibility Services.  SSAS is located in the new ACE space on the first floor of the Hewes Library, opposite Einstein’s Bros Bagels. They can be reached at 309-457-2257 or via email at: ssas@monmouthcollege.edu.


\section{Calendar}

\textit{This calendar is subject to change based on the circumstances of the course.}

\begin{center}
\begin{tabular}{llll}
\underline{Week} & \underline{Dates} & \underline{Notes} & \underline{Assignments Due} \\
1 & 1/16 --- 1/17 &  & \\
2 & 1/20 --- 1/24 &  & \\
3 & 1/27 --- 1/31 &  & Project 1 Checkpoint\\
4 & 2/3 --- 2/7 & &  Reviews Due\\
5 & 2/10 --- 2/14 & &  Project 1 Final  \\
6 & 2/17 --- 2/21 & &   Reviews Due \\
7 & 2/24 --- 2/28 & &  \\
8 & 3/2 --- 3/5 & &  Project 2 Checkpoint   \\
& 3/9 --- 3/13 &  SPRING BREAK & \\
9 & 3/16 --- 3/20 &  & Review Due \\
10 & 3/23 --- 3/27 &   & Project 2 Final \\
11 & 3/30 --- 4/3 & & Review Due\\
12 & 4/6 --- 4/10 & EASTER (M)  &  \\
13 & 4/13 --- 4/17 & EASTER(F) & Project 3 Checkpoint \\
14 & 4/20 --- 4/24 & SCHOLAR'S DAY (Tu) & Reviews Due  \\
15 & 4/27 --- 5/1 &  &  \\
16 & 5/4 --- 5/8 & READING DAY (Th)  & Project 3 Final \\
Final's Week & 5/12 & 6:30-9:30pm & Reviews Due. Self-Evaluation \\
\end{tabular}
\end{center}

\end{document}
